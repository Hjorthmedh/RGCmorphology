\documentclass[11pt]{article}

\usepackage{amsmath}
\usepackage{amssymb}

\usepackage{mathpazo}

\usepackage{graphicx}

\usepackage{cite}

\usepackage{color} 

% Use doublespacing - comment out for single spacing
\usepackage{setspace} 
\onehalfspacing
%\doublespacing

\usepackage[utf8]{inputenc} 

\usepackage{dcolumn}
\newcolumntype{.}{D{.}{.}{-1}}

\usepackage{marginnote}
%%\newcommand{\mycomment}[1]{\marginnote{#1}}
\newcommand{\mycomment}[1]{}
% use \mycomment{comment text}

% Boolean expressions
% \usepackage{etoolbox}
%
% \newtoggle{groupResults}
% \toggletrue{groupResults}
% \togglefalse{groupResults}


\usepackage[a4paper,margin=3cm]{geometry}
\usepackage{url}
% Bold the 'Figure #' in the caption and separate it with a period
% Captions will be left justified
\usepackage[labelfont=bf,labelsep=period,justification=raggedright]{caption}

% Get formating for table
\usepackage{siunitx}

\usepackage{booktabs}

\usepackage{lastpage}

% Rotated table
\usepackage{rotating}


% J Neurosci stylesheet - not working!
\usepackage{namedplus}
\bibliographystyle{namedplus}
% Use the PLoS provided bibtex style
%\bibliographystyle{plos2009}
% \bibliographystyle{modelcomp}

% Remove brackets from numbering in List of References
% \makeatletter
% \renewcommand{\@biblabel}[1]{\quad#1.}
% \makeatother

% Space after your macros, latex being obnoxious
\usepackage{xspace}

\usepackage{multirow}

% Turn of hyphenation to make word count more accurate
% \usepackage[none]{hyphenat} 
\tolerance=1
\emergencystretch=\maxdimen
\hyphenpenalty=10000
\hbadness=10000
\sloppy

% Leave date blank
\date{}

\pagestyle{myheadings}
%% ** EDIT HERE **

\newcommand{\Sumbul}{S\"{u}mb\"{u}l\xspace}


%% ** EDIT HERE **
%% PLEASE INCLUDE ALL MACROS BELOW

\DeclareMathOperator*{\argmin}{arg\,min}

% pedex and apex, subscript and superscript

\providecommand*{\ped}[1]{%
\ensuremath{_\mathrm{#1}}}
\providecommand*{\ap}[1]{%
\ensuremath{^\mathrm{#1}}}
\newcommand{\gMathV}{\textit{Math5}$\ap{\mathrm{-/-}}$\xspace}
\newcommand{\pMathV}{Math5$\ap{\mathrm{-/-}}$\xspace}

\newcommand{\gIsl}{\textit{Isl2}\xspace}
\newcommand{\gIslE}{\textit{Isl2-EphA3}\xspace}
\newcommand{\gIslEkk}{\textit{Isl2-EphA3}$\ap{\mathrm{ki/ki}}$\xspace}
\newcommand{\gIslEkp}{\textit{Isl2-EphA3}$\ap{\mathrm{ki/+}}$\xspace}

\newcommand{\gIslp}{\textrm{Isl2\ap{+}}\xspace}
\newcommand{\gIslm}{\textrm{Isl2\ap{-}}\xspace}

\newcommand{\gTKO}{\textit{{ephrin-A2,A3,A5}}\xspace}
\newcommand{\gephrinA}{\textit{ephrin-A}\xspace}
\newcommand{\gEphA}{\textit{EphA}\xspace}

\newcommand{\gEphAIII}{\textit{EphA3}\xspace}
\newcommand{\gEphAIV}{\textit{EphA4}\xspace}
\newcommand{\gEphAV}{\textit{EphA5}\xspace}

\usepackage{upgreek}
%\newcommand{\betaIIKO}{\ensuremath{\upbeta 2^{-/-}}\xspace}
\newcommand{\betaIIKO}{\ensuremath{\beta\mathit{2}^{-/-}}\xspace}

%% END MACROS SECTION

\begin{document}

\singlespacing


% \begin{table}
% \begin{tabular}{p{4.5cm}p{7cm}p{5cm}}
%   \toprule
%   Genetic type & Description & Citation \\
%   \midrule
%   \footnotesize{Calretinin (CB2)} & \footnotesize{Transient OFF-alpha}  & \footnotesize{\citetext{Huberman2008}}\\
%   \footnotesize{Cadherin-3 (Cdh3)} & \footnotesize{M2 ipRGCs \& "diving cells"} & \footnotesize{\citetext{Osterhout2011}}\\
%   \footnotesize{Dopamine Receptor 4 (DRD4)} & \footnotesize{ON-OFF direction selective (posterior motion)} & \footnotesize{\citetext{Huberman2009}}\\
%   \footnotesize{Homeobox d10 (Hoxd10)} & \footnotesize{On-DS (3 types) and On-Off DS/anterior tuned (1 type)} & \footnotesize{\citetext{Dhande2013}}\\
%   \footnotesize{Thyrotropin-Releasing Hormone Receptor (TRHR)} & \footnotesize{ON-OFF direction selective (posterior motion)}& \footnotesize{\citetext{RivlinEtzion2011}}\\
%   \bottomrule
% \end{tabular}
% \caption{Description of the five genetic types in this study.}
% \label{tab:geneticTypes}
% \end{table}

\doublespacing

\begin{table}
\begin{tabular}{p{4.5cm} p{4cm} l}
  \toprule
  Genetic type & Description & Citation \\
  \midrule
  \footnotesize{Calretinin (CB2)} & \footnotesize{Transient Off-alpha}  & \footnotesize{\citetext{Huberman2008}}\\
  \footnotesize{Cadherin-3 (Cdh3)} & \footnotesize{M2 ipRGCs \& "diving cells"} & \footnotesize{\citetext{Osterhout2011}}\\
  \footnotesize{Dopamine Receptor 4 (DRD4)} & \footnotesize{On-Off direction selective (posterior motion)} & \footnotesize{\citetext{Huberman2009}}\\
  \footnotesize{Homeobox d10 (Hoxd10)} & \footnotesize{On-DS (3 types) and On-Off DS/anterior tuned (1 type)} & \footnotesize{\citetext{Dhande2013}}\\
  \footnotesize{Thyrotropin-Releasing Hormone Receptor (TRHR)} & \footnotesize{On-Off direction selective (posterior motion)}& \footnotesize{\citetext{RivlinEtzion2011}}\\
  \bottomrule
\end{tabular}
\caption{Description of the five genetic types in this study.}
\label{tab:geneticTypes}
\end{table}


\clearpage

\singlespacing

\begin{table}
  {\renewcommand{\arraystretch}{1.3} \begin{tabular}{llp{8cm}}
      \toprule
      \textbf{Feature Name} & \textbf{Abbrev.} & \textbf{Description} \\

      \midrule
      Bistratification Distance &  BD & {\footnotesize Distance between putative bistratified dendrites (normalised by variance). See Methods.}\\
      Branch Asymmetry & BA & {\footnotesize Ratio of number of leaves between a branch points branches.}\\
      Dendritic Area & DA & {\footnotesize Area of dendritic arbor projected on the XY-plane, calculated using the convex hull}\\
      Dendritic Density & DD & {\footnotesize TDL / DA}\\
      Dendritic Diameter & DDi & {\footnotesize  Largest distance between two points on the dendritic arbor}\\
      Density of Branch Points & DBP &  {\footnotesize  NBP / DA }\\
      Fractal Dimension & FD & {\footnotesize  Dendritic complexity measure, see Methods}\\
      Mean Branch Angle & MBA &  {\footnotesize  Mean angle between the branches in 3D space.}\\
      Mean Segment Length & MSL & {\footnotesize  Mean length of dendritic segments}\\
      Mean Segment Tortuosity & MST & {\footnotesize  Ratio of segment length to distance between end points}\\
      Mean Terminal Segment Length & MTSL & {\footnotesize  Mean length of the last segment of each dendrite}\\
      Number of Branch Points & NBP &  {\footnotesize  Number of points where the dendrites branch}\\
      Soma Area & SA & {\footnotesize  Area of the soma projected onto the XY-plane}\\
      Stratification Depth & SD & {\footnotesize Dendritic centre of mass along z-axis, with soma at $z=0$}\\
      Total Dendritic Length & TDL & {\footnotesize Total length of all dendrites}\\

  \bottomrule
\end{tabular}
}
\caption{Features calculated from each RGC.}
\label{tab:featurelist}
\end{table}




\clearpage

%%%%%%%%%%%%%%%%%%%%%%%%%%%%%%%%%%%%%%%%%%%%%%%%%%%%%%%%%%%%%%%%%%%%%%%%%%%%
%
% Generated by featureValueTable.m
%

\begin{sidewaystable}
\begin{tabular}{llllll}
\toprule
 & \multicolumn{5}{c}{Genetic type}\\
% \cline{2-6}
Feature name & CB2 & Cdh3 & DRD4 & Hoxd10 & TRHR\\
\midrule
Bistratification Distance ($\mu m$)& $1.5 \pm 1.1$& $3.7 \pm 4.1$& $2.3 \pm 1.1$& $1.9 \pm 2.1$& $3.2 \pm 2.1$\\
Branch Assymetry& $0.63 \pm 0.02$& $0.74 \pm 0.02$& $0.71 \pm 0.03$& $0.67 \pm 0.03$& $0.71 \pm 0.03$\\
Dendritic Area ($mm^2$)& $0.06 \pm 0.02$& $0.04 \pm 0.01$& $0.05 \pm 0.01$& $0.07 \pm 0.04$& $0.03 \pm 0.01$\\
Dendritic Density ($\mu m^{-1}$)& $0.07 \pm 0.01$& $0.12 \pm 0.03$& $0.13 \pm 0.02$& $0.08 \pm 0.03$& $0.16 \pm 0.02$\\
Dendritic Diameter ($\mu m$)& $326 \pm 75$& $270 \pm 52$& $300 \pm 37$& $358 \pm 104$& $248 \pm 34$\\
Density of Branch Points ($mm^{-2}$)& $1351 \pm 675$& $7697 \pm 2758$& $5235 \pm 1031$& $2295 \pm 1785$& $7150 \pm 1466$\\
Fractal Dimension Box Counting& $1.40 \pm 0.02$& $1.47 \pm 0.04$& $1.49 \pm 0.03$& $1.44 \pm 0.06$& $1.54 \pm 0.04$\\
Mean Branch Angle (degrees)& $100 \pm 4$& $99 \pm 5$& $104 \pm 3$& $103 \pm 4$& $102 \pm 3$\\
Mean Segment Length ($\mu m$)& $29.5 \pm 7.1$& $7.9 \pm 1.5$& $12.7 \pm 2.4$& $22.3 \pm 7.9$& $11.2 \pm 2.2$\\
Mean Segment Tortuosity& $1.15 \pm 0.02$& $1.21 \pm 0.05$& $1.20 \pm 0.05$& $1.20 \pm 0.07$& $1.17 \pm 0.03$\\
Mean Terminal Segment Length ($\mu m$)& $36.3 \pm 10.1$& $6.3 \pm 1.2$& $11.8 \pm 2.7$& $20.9 \pm 8.9$& $10.0 \pm 3.0$\\
Number of Branch Points& $70 \pm 12$& $280 \pm 87$& $242 \pm 66$& $124 \pm 47$& $233 \pm 69$\\
Soma Area ($\mu m^2$)& $348 \pm 92$& $141 \pm 34$& $180 \pm 52$& $202 \pm 74$& $190 \pm 51$\\
Stratification Depth ($\mu m$)& $-7 \pm 3$& $-10 \pm 3$& $-11 \pm 4$& $-11 \pm 4$& $-10 \pm 4$\\
Total Dendritic Length ($mm$)& $4.3 \pm 1.2$& $4.4 \pm 1.0$& $6.0 \pm 1.1$& $5.2 \pm 1.6$& $5.1 \pm 0.9$\\
\bottomrule
\end{tabular}
\caption{Mean and standard deviation of features for each genetic type.}
\label{tab:featVals}
\end{sidewaystable}


\clearpage

%%%%%%%%%%%%%%%%%%%%%%%%%%%%%%%%%%%%%%%%%%%%%%%%%%%%%%%%%%%%%%%%%%%%%%%%%%%%

% r = RGCclass(0);
% r.lazyLoad();
% r.featureCorrelation()


\begin{sidewaystable}
\begin{tabular}{lrrrrrrrrrrrrrrr}
\toprule
& BD& BA& DA& DD& DDi& DBP& FD& MBA& MSL& MST& MTSL& NBP& SA& SD& TDL\\
\midrule
Bistratification Distance (BD) & \textbf{}  &  &  &  &  &  &  &  &  &  &  &  &  &  & \\
Branch Assymetry (BA) & 0.25 & \textbf{}  &  &  &  &  &  &  &  &  &  &  &  &  & \\
Dendritic Area (DA) & -0.37 & -0.27 & \textbf{}  &  &  &  &  &  &  &  &  &  &  &  & \\
Dendritic Density (DD) & \textbf{0.47} & \textbf{0.52} & \textbf{-0.74} & \textbf{}  &  &  &  &  &  &  &  &  &  &  & \\
Dendritic Diameter (DDi) & -0.37 & -0.20 & \textbf{0.96} & \textbf{-0.69} & \textbf{}  &  &  &  &  &  &  &  &  &  & \\
Density of Branch Points (DBP) & \textbf{0.51} & \textbf{0.79} & \textbf{-0.63} & \textbf{0.83} & \textbf{-0.60} & \textbf{}  &  &  &  &  &  &  &  &  & \\
Fractal Dimension (FD) & 0.38 & \textbf{0.50} & \textbf{-0.52} & \textbf{0.84} & \textbf{-0.47} & \textbf{0.70} & \textbf{}  &  &  &  &  &  &  &  & \\
Mean Branch Angle (MBA) & 0.13 & -0.02 & 0.02 & 0.13 & 0.06 & -0.07 & -0.00 & \textbf{}  &  &  &  &  &  &  & \\
Mean Segment Length (MSL) & -0.36 & \textbf{-0.81} & \textbf{0.65} & \textbf{-0.73} & \textbf{0.60} & \textbf{-0.85} & \textbf{-0.64} & -0.01 & \textbf{}  &  &  &  &  &  & \\
Mean Segment Tortuosity (MST) & 0.16 & 0.10 & -0.06 & 0.24 & -0.03 & 0.11 & 0.05 & 0.29 & -0.07 & \textbf{}  &  &  &  &  & \\
Mean Terminal Segment Length (MTSL) & -0.31 & \textbf{-0.80} & \textbf{0.52} & \textbf{-0.65} & \textbf{0.49} & \textbf{-0.78} & \textbf{-0.59} & -0.05 & \textbf{0.97} & -0.13 & \textbf{}  &  &  &  & \\
Number of Branch Points (NBP) & 0.32 & \textbf{0.90} & -0.25 & \textbf{0.60} & -0.18 & \textbf{0.81} & \textbf{0.60} & -0.03 & \textbf{-0.78} & 0.09 & \textbf{-0.75} & \textbf{}  &  &  & \\
Soma Area (SA) & -0.15 & \textbf{-0.44} & 0.21 & -0.32 & 0.19 & -0.39 & -0.21 & -0.26 & \textbf{0.44} & \textbf{-0.47} & \textbf{0.52} & -0.35 & \textbf{}  &  & \\
Stratification Depth (SD) & -0.13 & -0.04 & 0.28 & -0.32 & 0.27 & -0.13 & -0.24 & -0.10 & 0.23 & \textbf{-0.44} & 0.25 & 0.04 & 0.22 & \textbf{}  & \\
Total Dendritic Length (TDL) & -0.09 & 0.27 & \textbf{0.58} & 0.02 & \textbf{0.63} & -0.02 & 0.21 & 0.18 & 0.06 & 0.15 & -0.01 & \textbf{0.45} & -0.02 & 0.12 & \textbf{} \\
\bottomrule
\end{tabular}
\caption{Correlation coefficient calculated for pairs of 15
  features. Features which are highly correlated or anti-correlated
  are marked in bold.}
\label{tab:corr}\end{sidewaystable}





% % blindClusteringBatch.m

% % SAVES files in RESULTS
% % from: RESULTS/BlindClustering-latex-n-5.tex

% \begin{table}
% \begin{tabular}{llrrrrr}
% \toprule
%  & & \multicolumn{5}{c}{Blind Cluster}\\
% \cline{3-7}
%  & & 1 & 2 & 3 & 4 & 5\\
% \midrule
% \multirow{5}{*}{\rotatebox{90}{Genetic Type}}& CB2 & 0 & 0 & 0 & 19 & 0\\
% & Cdh3 & 7 & 1 & 3 & 0 & 0\\
% & DRD4 & 13 & 13 & 0 & 0 & 0\\
% & Hoxd10 & 3 & 9 & 0 & 3 & 14\\
% & TRHR & 6 & 2 & 1 & 0 & 0\\
% \bottomrule %\cline{3-7}
% \end{tabular}
% \caption{Comparison of blind clustering and genetic type. This table
%   was done using our uncorrelated feature set.}
% \label{tab:blind5confusion}
% \end{table}



% blindClusteringBatch.m

% SAVES files in RESULTS
% from: RESULTS/BlindClustering-latex-n-5.tex

% This table uses our 5-feature set
\begin{table}
\begin{tabular}{llrrrrr}
\hline
 & & \multicolumn{5}{c}{Blind cluster}\\
 & & 1 & 2 & 3 & 4 & 5\\
\cline{3-7}
\multirow{5}{*}{\rotatebox{90}{Genetic type}}& CB2 & 0 & 11 & 0 & 0 & 8\\
& Cdh3 & 0 & 0 & 4 & 7 & 0\\
& DRD4 & 0 & 0 & 20 & 6 & 0\\
& Hoxd10 & 12 & 2 & 12 & 2 & 1\\
& TRHR & 0 & 0 & 0 & 9 & 0\\
\cline{3-7}
\end{tabular}
\caption{Comparison of blind clustering with genetic type.}
\label{tab:blind5confusion}
\end{table}



% These use our 3-feature-set classifier
%
% \begin{table}
% \begin{tabular}{lrr}
%  & 1 & 2\\
% \hline
% CB2 & 0 & 19\\
% Cdh3 & 11 & 0\\
% DRD4 & 26 & 0\\
% Hoxd10 & 15 & 14\\
% TRHR & 9 & 0\\
% \hline
% \end{tabular}
% \caption{Comparison of blind clustering and genetic labeling. Columns shows two blindly generated clusters, and rows the corresponding cells geneticl labeling.}
% \label{tab:blindconfusion2}
% \end{table}

% \clearpage

% \begin{table}
% \begin{tabular}{lrrrrr}
%  & 1 & 2 & 3 & 4 & 5\\
% \hline
% CB2 & 0 & 0 & 4 & 0 & 15\\
% Cdh3 & 2 & 0 & 0 & 9 & 0\\
% DRD4 & 1 & 7 & 0 & 18 & 0\\
% Hoxd10 & 15 & 4 & 1 & 5 & 4\\
% TRHR & 0 & 6 & 0 & 3 & 0\\
% \hline
% \end{tabular}
% \caption{Comparison of blind clustering and genetic labeling. Columns shows the five blindly generated clusters, and rows the corresponding cells geneticl labeling.}
% \label{tab:blindconfusion5}
% \end{table}


\clearpage

% Use r.benchmark(1000) to generate statistics
% Update: doBenchmark

% \begin{table}
% \begin{tabular}{lll}
% \textbf{Method} & \textbf{Features} & \textbf{Performance} \\
% \hline

% Na\"{i}ve Bayes & DD, SA, DF, BSD & $69.4 \pm 2.8\,\%$\\
% Na\"{i}ve Bayes & DD, SA, FD, BA & $71.1 \pm 2.8\,\%$\\
% Na\"{i}ve Bayes & DD, SA, FD, DF & $68.4 \pm 2.7\,\%$\\
% Na\"{i}ve Bayes & DD, SA, TDL, TSL & $80.3 \pm 2.9\,\%$ \\
% Na\"{i}ve Bayes & DBP, SA, DF, TSL & $ 77.3 \pm 2.8\,\%$ \\ % 4-tuple, exhaustive search, features scored individually
% Na\"{i}ve Bayes & DBP, SA, FD, TSL & $ 79.7 \pm 2.6\,\%$ \\ % 4-tuple, exhaustive search, individually scored from top 30
% Na\"{i}ve Bayes & PCA& $ 75.7 \pm 2.5\,\%$ \\ % First 4 PCA components, derived from all 16
% Na\"{i}ve Bayes & \textbf{FD, TSL, SA} & $ 78.7 \pm 2.2\,\%$ \\ % 3 tuple that was best (redo with more iter)
% Na\"{i}ve Bayes & \textbf{DF, DD, SD} & $ 61.5 \pm 2.6\,\%$ \\ % Feature set similar to what is used by Kong et al 2005

% Bags & DD, SA, DF, BSD & $ 66.1 \pm 2.7\,\%$\\
% Bags & DD, SA, FD, BA & $ 67.7 \pm 2.9\,\%$\\
% Bags & DD, SA, FD, DF & $  71.3 \pm 2.6\,\%$ \\ 
% Bags & DD, SA, TDL, TSL & $ 78.5 \pm 2.5\,\%$\\
% Bags & DBP, SA, DF, TSL & $ 74.6 \pm 2.5\,\%$ \\
% Bags & DBP, SA, FD, TSL & $ 77.0 \pm 2.2\,\%$ \\

% % !!! Add Bags and a few more here
% \hline
% \end{tabular}


% runExhaustiveFeatureSearch     % !!! This is slow
% analyseExhaustiveFeatureSearch
% 

% 20 reps
\begin{sidewaystable}
\begin{tabular}{cclllllllllllllll}
Number of features & Performance  & BD & BA & DA & DD & DDi & DBP & FD & MBA & MSL & MST & MTSL & NBP & SA & SD & TDL\\
\hline
1 & $64.7 \pm 1.7\,\%$  &  &  &  &  &  &  &  &  &  &  & $\bullet$ &  &  &  & \\
2 & $72.9 \pm 2.0\,\%$  &  &  &  &  &  & $\bullet$ &  &  &  &  &  &  & $\bullet$ &  & \\
3 & $78.7 \pm 2.8\,\%$  &  &  &  &  &  &  & $\bullet$ &  &  &  & $\bullet$ &  & $\bullet$ &  & \\
4 & $80.7 \pm 3.1\,\%$  &  &  & $\bullet$ &  &  &  & $\bullet$ &  &  &  & $\bullet$ &  & $\bullet$ &  & \\
5 & $83.1 \pm 3.6\,\%$  &  &  & $\bullet$ &  &  & $\bullet$ & $\bullet$ &  &  &  & $\bullet$ &  & $\bullet$ &  & \\
6 & $84.0 \pm 2.7\,\%$  &  &  & $\bullet$ &  &  &  & $\bullet$ &  & $\bullet$ & $\bullet$ &  & $\bullet$ & $\bullet$ &  & \\
7 & $85.5 \pm 2.9\,\%$  &  &  & $\bullet$ &  &  & $\bullet$ & $\bullet$ &  &  & $\bullet$ & $\bullet$ & $\bullet$ & $\bullet$ &  & \\
8 & $86.3 \pm 2.6\,\%$  &  &  & $\bullet$ &  &  & $\bullet$ & $\bullet$ & $\bullet$ &  & $\bullet$ & $\bullet$ & $\bullet$ & $\bullet$ &  & \\
9 & $85.7 \pm 2.1\,\%$  & $\bullet$ &  & $\bullet$ & $\bullet$ &  &  & $\bullet$ &  &  & $\bullet$ & $\bullet$ & $\bullet$ & $\bullet$ &  & $\bullet$\\
10 & $85.1 \pm 2.2\,\%$  &  &  & $\bullet$ &  &  & $\bullet$ & $\bullet$ & $\bullet$ & $\bullet$ & $\bullet$ & $\bullet$ & $\bullet$ & $\bullet$ &  & $\bullet$\\
11 & $85.2 \pm 1.7\,\%$  &  &  & $\bullet$ & $\bullet$ & $\bullet$ & $\bullet$ & $\bullet$ &  & $\bullet$ & $\bullet$ & $\bullet$ & $\bullet$ & $\bullet$ &  & $\bullet$\\
12 & $84.9 \pm 1.7\,\%$  &  &  & $\bullet$ & $\bullet$ & $\bullet$ & $\bullet$ & $\bullet$ & $\bullet$ & $\bullet$ & $\bullet$ & $\bullet$ & $\bullet$ & $\bullet$ &  & $\bullet$\\
13 & $84.1 \pm 1.7\,\%$  &  & $\bullet$ & $\bullet$ & $\bullet$ & $\bullet$ & $\bullet$ & $\bullet$ & $\bullet$ & $\bullet$ & $\bullet$ & $\bullet$ & $\bullet$ & $\bullet$ &  & $\bullet$\\
14 & $83.2 \pm 2.0\,\%$  & $\bullet$ & $\bullet$ & $\bullet$ & $\bullet$ & $\bullet$ & $\bullet$ & $\bullet$ & $\bullet$ & $\bullet$ & $\bullet$ & $\bullet$ & $\bullet$ & $\bullet$ &  & $\bullet$\\
15 & $82.2 \pm 1.8\,\%$  & $\bullet$ & $\bullet$ & $\bullet$ & $\bullet$ & $\bullet$ & $\bullet$ & $\bullet$ & $\bullet$ & $\bullet$ & $\bullet$ & $\bullet$ & $\bullet$ & $\bullet$ & $\bullet$ & $\bullet$\\
\bottomrule
\end{tabular}
\caption{Performance of classifiers. Five-fold cross-validation,
  repeated twenty times for each feature set. Bistratification
  Distance (BD), Branch Assymetry (BA), Dendritic Area (DA), Dendritic
  Density (DD), Dendritic Diameter (DDi), Density of Branch Points
  (DBP), Fractal Dimension (FD), Mean Branch Angle
  (MBA), Mean Segment Length (MSL), Mean Segment Tortuosity (MST),
  Mean Terminal Segment Length (MTSL), Number of Branch Points (NBP),
  Soma Area (SA), Stratification Depth (SD), Total Dendritic Length
  (TDL). Performance is given as mean $\pm$ standard deviation
    correctly classified.}
\label{tab:performance}
\end{sidewaystable}



\clearpage

% getConfusionMatrixLatex

% !!! OBS, need to make diagonal elements bold by hand! Since the general latex generating function I wrote
% does not allow for that, it would affect other tables alos.

\begin{table}
\begin{tabular}{llrrrrr}
\toprule
 & & \multicolumn{5}{c}{Predicted type}\\
\cline{3-7}
 & & CB2 & Cdh3 & DRD4 & Hoxd10 & TRHR\\
\midrule
\multirow{5}{*}{\rotatebox{90}{Genetic type}}& CB2 & \textbf{18} & 0 & 0 & 1 & 0\\
& Cdh3 & 0 & \textbf{9} & 0 & 1 & 1\\
& DRD4 & 0 & 0 & \textbf{25} & 0 & 1\\
& Hoxd10 & 2 & 1 & 1 & \textbf{23} & 2\\
& TRHR & 0 & 1 & 3 & 0 & \textbf{5}\\
\bottomrule % \cline{3-7}
\end{tabular}
\caption{Confusion matrix result.}
\label{tab:confusionMatrixLeaveOneOut}
\end{table}

\clearpage

%
% Removing TRHR from the data set
% Our old feature set increased from about 80% to almost 85%.
%  fractalDimensionBoxCounting, meanTerminalSegmentLength, somaArea
%
%
% 1. 0.850 +/- 0.023 : dendriticField, numLeaves, somaArea
% 2. 0.847 +/- 0.024 : dendriticField, numSegments, somaArea
% 3. 0.846 +/- 0.024 : dendriticField, numBranchPoints, somaArea
% 4. 0.845 +/- 0.022 : fractalDimensionBoxCounting, meanTerminalSegmentLength, somaArea
% 5. 0.841 +/- 0.022 : densityOfBranchPoints, numLeaves, somaArea
% 6. 0.840 +/- 0.022 : densityOfBranchPoints, numSegments, somaArea
% 7. 0.839 +/- 0.022 : densityOfBranchPoints, numBranchPoints, somaArea
% 8. 0.831 +/- 0.020 : densityOfBranchPoints, meanTerminalSegmentLength, somaArea
% 9. 0.829 +/- 0.025 : densityOfBranchPoints, somaArea, totalDendriticLength
% 10. 0.829 +/- 0.021 : branchAssymetry, densityOfBranchPoints, somaArea
% 11. 0.829 +/- 0.023 : densityOfBranchPoints, meanSegmentLength, somaArea
% 12. 0.829 +/- 0.024 : meanTerminalSegmentLength, numBranchPoints, somaArea
% 13. 0.826 +/- 0.023 : meanTerminalSegmentLength, numSegments, somaArea
% 14. 0.825 +/- 0.027 : dendriticDensity, meanTerminalSegmentLength, somaArea
% 15. 0.824 +/- 0.025 : meanTerminalSegmentLength, somaArea, totalDendriticLength
% 16. 0.824 +/- 0.024 : meanTerminalSegmentLength, numLeaves, somaArea
% 17. 0.823 +/- 0.022 : meanSegmentLength, numBranchPoints, somaArea
% 18. 0.822 +/- 0.023 : meanSegmentLength, numSegments, somaArea
% 19. 0.821 +/- 0.021 : meanSegmentTortuosity, meanTerminalSegmentLength, somaArea
% 20. 0.821 +/- 0.026 : dendriticDensity, meanTerminalSegmentLength, totalDendriticLength
% 21. 0.820 +/- 0.023 : meanSegmentLength, numLeaves, somaArea
% 22. 0.819 +/- 0.027 : dendriticField, fractalDimensionBoxCounting, meanTerminalSegmentLength
% 23. 0.818 +/- 0.020 : dendriticDensity, numLeaves, somaArea
% 24. 0.817 +/- 0.020 : dendriticDensity, numSegments, somaArea
% 25. 0.815 +/- 0.024 : biStratificationDistance, numLeaves, somaArea
% 26. 0.815 +/- 0.024 : biStratificationDistance, numSegments, somaArea
% 27. 0.814 +/- 0.021 : dendriticDensity, numBranchPoints, somaArea
% 28. 0.814 +/- 0.024 : biStratificationDistance, numBranchPoints, somaArea
% 29. 0.812 +/- 0.023 : dendriticField, densityOfBranchPoints, meanTerminalSegmentLength
% 30. 0.812 +/- 0.022 : biStratificationDistance, meanTerminalSegmentLength, somaArea
% Score board, feature importance weighted
% 1. Score: 26.921 : meanTerminalSegmentLength
% 2. Score: 26.479 : densityOfBranchPoints
% 3. Score: 26.223 : numBranchPoints
% 4. Score: 26.220 : numSegments
% 5. Score: 26.209 : numLeaves
% 6. Score: 26.089 : meanSegmentLength
% 7. Score: 26.076 : somaArea
% 8. Score: 25.306 : dendriticField
% 9. Score: 25.167 : branchAssymetry
% 10. Score: 24.967 : dendriticDensity
% 11. Score: 24.720 : meanSegmentTortuosity
% 12. Score: 24.571 : totalDendriticLength
% 13. Score: 24.515 : meanBranchAngle
% 14. Score: 24.341 : fractalDimensionBoxCounting
% 15. Score: 24.167 : biStratificationDistance
% 16. Score: 23.560 : stratificationDepth
% Score board (only from top 30 entries), feature importance weighted
% 1. Score: 6.120 : somaArea
% 2. Score: 2.636 : meanTerminalSegmentLength
% 3. Score: 1.870 : densityOfBranchPoints
% 4. Score: 1.423 : numLeaves
% 5. Score: 1.332 : numSegments
% 6. Score: 1.331 : numBranchPoints
% 7. Score: 1.164 : dendriticField
% 8. Score: 1.136 : dendriticDensity
% 9. Score: 0.884 : meanSegmentLength
% 10. Score: 0.851 : biStratificationDistance
% 11. Score: 0.655 : totalDendriticLength
% 12. Score: 0.474 : fractalDimensionBoxCounting
% 13. Score: 0.243 : branchAssymetry
% 14. Score: 0.228 : meanSegmentTortuosity
% 15. Score: 0.000 : meanBranchAngle
% 16. Score: 0.000 : stratificationDepth
% Wrote data to chooseFeature-results-735783.64439.mat
% Elapsed time is 13188.359832 seconds.


\clearpage

% Use blindClusterBatch.m
% But you need to change features manually, there is an if(0) statement

% \begin{table}
% \begin{tabular}{llrr}
%  & & \multicolumn{2}{c}{Blind Cluster}\\
%  & & 1 & 2\\
% \cline{3-4}
% \multirow{5}{*}{\rotatebox{90}{Genetic Type}}& CB2 & 0 & 19\\
% & Cdh3 & 11 & 0\\
% & DRD4 & 26 & 0\\
% & Hoxd10 & 14 & 15\\
% & TRHR & 9 & 0\\
% \cline{3-4}
% \end{tabular}

% \vspace{1cm}


% \begin{tabular}{llrrr}
%  & & \multicolumn{3}{c}{Blind Cluster}\\
%  & & 1 & 2 & 3\\
% \cline{3-5}
% \multirow{5}{*}{\rotatebox{90}{Genetic Type}}& CB2 & 19 & 0 & 0\\
% & Cdh3 & 0 & 0 & 11\\
% & DRD4 & 0 & 0 & 26\\
% & Hoxd10 & 2 & 15 & 12\\
% & TRHR & 0 & 0 & 9\\
% \cline{3-5}
% \end{tabular}
% \caption{Blind clustering using the best 5-feature set used
%   for the supervised classification. }
% \label{tab:blindOracleFeatures}
% \end{table}


% \clearpage



% \section*{Old Results Section}

% Using data from genetically labelled RGC we have built a supervised
% classifier to identify RGC types using morphological features.

% Hierarchical clustering on the feature vectors indicated that a large
% subset of the features were correlated. This reduction of features
% from 16 down to 9, for comparison see Table~\ref{tab:corr} where
% features with significant correlation are marked in bold.  We then
% used Matlab's built in function \texttt{sequentialfs} to select
% features to use for the classification. The function sequentially adds
% one feature at a time, and it can be used to generate a graph showing
% how the classification error first sharply decreases with the first
% 3-4 features, reaches a minima at around 9 features, and then
% increases as the feature set is increased to 16. The performance of
% Na\"{i}ve Bayes was overall better than the other classification
% methods, with Bags being the runner up. Guided by thie sequential
% feature selection we did an exhaustive search for all 3-tuples and
% 4-tuples of features. Adding a fourth feature does improve the
% performance somewhat, but there were no clear winner combination,
% between runs different sets of features came out the best. The best
% 3-tuple performed almost as well (Table~\ref{tab:performance}), so we
% decided to use Soma Area, Mean Terminal Segment Length and Fractal
% Dimension Box Counting. The last measure looks at the dendritic
% coverage of a grid at different length scales (See Methods).




% Genearte the landscape figures
% r = RGCclass(0); r.lazyLoad();
% r.featureSelection();

% Exhaustive test of all 4-tuples.
% r = RGCclass(0); r.lazyLoad();
% r.chooseFeatures();
%
% Additional analysis:
% chooseFeaturePostMortem
%
% Best 3-tuple: 78.7 +/- 2.2 fractalDimensionBoxCounting meanTerminalSegmentLength somaArea 


% Old scores below for 4-tuples:
%
% Scores after 200 repeats, using Naive Bayes classifiers
% Only using the non-correlated features
% 
% 1. Score: 8.901 : dendriticDensity
% 2. Score: 8.663 : fractalDimensionBoxCounting
% 3. Score: 8.621 : somaArea
% 4. Score: 8.620 : dendriticField
% 5. Score: 8.323 : meanSegmentTortuosity
% 6. Score: 8.318 : meanBranchAngle
% 7. Score: 8.184 : biStratificationDistance
% 8. Score: 8.171 : totalDendriticLength
% 9. Score: 7.992 : stratificationDepth

% Repeated, using all non-VAChT features
%
% !!! Running now...
% 2014-06-20 
%
% 1. 0.805 +/- 0.028 : fractalDimensionBoxCounting, meanTerminalSegmentLength, somaArea, totalDendriticLength
% 2. 0.801 +/- 0.029 : fractalDimensionBoxCounting, meanSegmentLength, somaArea, totalDendriticLength
% 3. 0.801 +/- 0.020 : densityOfBranchPoints, fractalDimensionBoxCounting, meanSegmentTortuosity, totalDendriticLength
% 4. 0.801 +/- 0.029 : dendriticDensity, meanTerminalSegmentLength, somaArea, totalDendriticLength
% 5. 0.800 +/- 0.025 : dendriticField, fractalDimensionBoxCounting, meanTerminalSegmentLength, somaArea
% 6. 0.799 +/- 0.026 : densityOfBranchPoints, fractalDimensionBoxCounting, meanTerminalSegmentLength, somaArea
% 7. 0.793 +/- 0.026 : biStratificationDistance, fractalDimensionBoxCounting, meanTerminalSegmentLength, somaArea
% 8. 0.789 +/- 0.023 : fractalDimensionBoxCounting, meanBranchAngle, meanTerminalSegmentLength, somaArea
% 9. 0.787 +/- 0.025 : densityOfBranchPoints, fractalDimensionBoxCounting, numBranchPoints, somaArea
% 10. 0.787 +/- 0.025 : dendriticDensity, meanSegmentTortuosity, meanTerminalSegmentLength, totalDendriticLength
% 11. 0.787 +/- 0.024 : densityOfBranchPoints, fractalDimensionBoxCounting, numSegments, somaArea
% 12. 0.786 +/- 0.025 : densityOfBranchPoints, fractalDimensionBoxCounting, numLeaves, somaArea
% 13. 0.786 +/- 0.022 : dendriticField, densityOfBranchPoints, meanSegmentTortuosity, meanTerminalSegmentLength
% 14. 0.782 +/- 0.026 : densityOfBranchPoints, numLeaves, somaArea, totalDendriticLength
% 15. 0.780 +/- 0.026 : dendriticDensity, fractalDimensionBoxCounting, meanTerminalSegmentLength, somaArea
% 16. 0.780 +/- 0.025 : densityOfBranchPoints, numSegments, somaArea, totalDendriticLength
% 17. 0.779 +/- 0.021 : densityOfBranchPoints, fractalDimensionBoxCounting, meanSegmentTortuosity, numBranchPoints
% 18. 0.778 +/- 0.024 : densityOfBranchPoints, numBranchPoints, somaArea, totalDendriticLength
% 19. 0.778 +/- 0.024 : dendriticDensity, meanBranchAngle, meanTerminalSegmentLength, totalDendriticLength
% 20. 0.778 +/- 0.029 : densityOfBranchPoints, meanTerminalSegmentLength, somaArea, totalDendriticLength
% Score board, feature importance weighted
% 1. Score: 81.121 : densityOfBranchPoints
% 2. Score: 80.976 : meanTerminalSegmentLength
% 3. Score: 80.306 : somaArea
% 4. Score: 80.016 : dendriticField
% 5. Score: 79.845 : fractalDimensionBoxCounting
% 6. Score: 79.526 : dendriticDensity
% 7. Score: 79.477 : meanSegmentLength
% 8. Score: 79.221 : numBranchPoints
% 9. Score: 79.177 : numSegments
% 10. Score: 79.114 : numLeaves
% 11. Score: 78.386 : meanSegmentTortuosity
% 12. Score: 78.184 : meanBranchAngle
% 13. Score: 78.005 : totalDendriticLength
% 14. Score: 77.293 : branchAssymetry
% 15. Score: 76.271 : biStratificationDistance
% 16. Score: 74.948 : stratificationDepth
% Elapsed time is 49387.996946 seconds.


% See: feature-selection-exhaustive.txt

% From the confusion matrix (Table~\ref{tab:confusion}) we learn that
% TRHR is particularly difficult to classify. On average half of the
% TRHR cells are misclassified as DRD4. One of them is also
% misclassified as Hoxd10. The CB2 cells form a distinct cluster
% without overlap with other classes. Cdh3, DRD4 and Hoxd10 cells have
% some overlap between the clusters but they are mostly correctly
% classified. Turning the table around, how sure are we a cell
% classified as a certain class is indeed of that class. Taking the data
% from Table~\ref{tab:confusion} we can calculate the true positive
% rates for each class. CB2: 90\,\%, Cdh3 77\,\%, DRD4 72\,\%, Hoxd10
% 89\,\% and TRHR 48\,\%.


% \subsubsection*{Blind Clustering}

% We performed blind clustering on the dataset using the k-means
% algorithm, and optimised the number of classes using the silhouette
% measure (k = 2 to 11). The silhouette measure was smallest for k = 2.

% Looking closer at the blind clustering using k=2: Cdh3, DRD4 and TRHR
% exclusively have members in the first cluster, CB2 only has members in
% the second cluster, and Hoxd10 is split evenly between the two
% clusters.

% When k is increased to 3, CB2 stays together in its own cluster. DRD4
% and TRHR mainly falls in a shared cluster, with a couple of their
% cells now instead clustering in a third cluster with 2/3 of the Hoxd10
% cells. The remaining Hoxd10 are mainly in the third cluster.

% Table~\ref{tab:blindconfusion5} compares blind clustering using five
% clusters to the genetical labeling. The first cluster is made up
% almost entirely of Hoxd10, and three cells genetically labeled as Cdh3
% or DRD4. The fifth cluster and the smaller third cluster are made up
% of mainly CB2 and a few Hoxd10 cells. Half the cells in the fourth
% cluster are DRD4, and the only cell type not represented in that
% cluster are CB2. The third cluster is roughly evenly divided between
% DRD4, Hoxd10 and TRHR.

% One important question is whether the genetic markers are distinct.
% Should any classes be split into two? To assess this we calculate the
% PCA components for each individual class.




% \begin{figure}[!ht]
% \begin{center}
% \includegraphics[width=15cm,trim=0 200 0 200, clip]{FIGS/Figure1/Clustering-based-on-feature-corrcoeff.pdf}
% % \includegraphics[width=6cm,trim=50 150 50 150, clip]{FIGS/Figure1/FeatureNumberErrorPlot-naiveBayes-nFeatures-16-seed-0-2014-05-16-11:27:10.pdf}

% \end{center}
% \caption{Feature selection. (A) Hierarchical clustering based on
%   pairwise correlation between feature vectors. Branches below 0.9
%   were pruned so that they only have one leaf, removed leaves are
%   marked in red.
% }
% \label{fig:hierarchical}
% \end{figure}


% rgc_pairs.R
%
% \begin{figure}[!ht]
% \begin{center}
% \includegraphics[width=15cm,trim=0 0 0 0, clip]{FIGS/FigurePairs/rgc_pairs.pdf}

% \end{center}
% \caption{Pairwise correlation.
% }
% \label{fig:pairwise}
% \end{figure}





% \begin{table}\begin{tabular}{llllll}
% & \bf{CB2} & \bf{Cdh3} & \bf{DRD4} & \bf{Hoxd10} & \bf{TRHR}\\
% \hline
% \textbf{CB2} & $17.71 \pm 0.78$  &  $0.00 \pm 0.00$  &  $0.00 \pm 0.03$  &  $1.28 \pm 0.78$   &  $0.00 \pm 0.00$\\
% \textbf{Cdh3} & $0.00 \pm 0.00$  &  $7.91 \pm 0.80$  &  $0.24 \pm 0.47$  &  $1.74 \pm 0.45$   &  $1.12 \pm 0.47$\\
% \textbf{DRD4} & $0.00 \pm 0.00$  &  $1.08 \pm 0.40$  &  $21.81 \pm 0.60$  &  $1.29 \pm 0.48$   &  $1.82 \pm 0.63$\\
% \textbf{Hoxd10} & $2.35 \pm 0.60$  &  $0.95 \pm 0.43$  &  $0.13 \pm 0.36$  &  $24.12 \pm 0.93$   &  $1.44 \pm 0.55$\\
% \textbf{TRHR} & $0.00 \pm 0.00$  &  $0.90 \pm 0.77$  &  $1.60 \pm 0.81$  &  $1.02 \pm 0.30$   &  $5.49 \pm 1.25$\\
% \end{tabular}
% \caption{Confusion matrix with cross validation: branchAssymetry
%   dendriticDensity dendriticDiameter dendriticField
%   densityOfBranchPoints fractalDimensionBoxCounting meanBranchAngle
%   meanTerminalSegmentLength numBranchPoints numSegments somaArea
%   totalDendriticLength. \textbf{Note that using these features in
%     5-fold cross validation only gives $81.9\pm 1.8\,\%$ correctness.}} 

% \end{table}

% CB2: 88.3 %
% Cdh3: 73.0 %
% DRD4: 91.7 %
% Hoxd10: 81.9 %
% TRHR: 55.6 %

% Using howManyRandomFeatureVectorsNeeded.m to test on synthetic data,
% if there is no structure in the feature vectors (uniform random
% numbers) then it is not possible to separate out the classes.

% \begin{table}
% \begin{tabular}{lrrrrr}
%  & CB2 & Cdh3 & DRD4 & Hoxd10 & TRHR\\
% \hline
% CB2 & 18 & 0 & 0 & 1 & 0\\
% Cdh3 & 0 & 8 & 0 & 2 & 1\\
% DRD4 & 0 & 1 & 22 & 1 & 2\\
% Hoxd10 & 2 & 1 & 0 & 24 & 2\\
% TRHR & 0 & 1 & 1 & 1 & 6\\
% \hline
% \end{tabular}
% \caption{Confusion matrix generated with leave one out: branchAssymetry dendriticDensity dendriticDiameter dendriticField densityOfBranchPoints fractalDimensionBoxCounting meanBranchAngle meanTerminalSegmentLength numBranchPoints numSegments somaArea totalDendriticLength 
% }
% \label{tab:confusionMatrixLeaveOneOut}
% \end{table}

% \clearpage

% \Sumbul data figures and tables following

% \begin{figure}[!ht]
% \begin{center}
% \includegraphics{FIGS/Sumbul/exhaustive-search-box-plot-Sumbul.eps}
% \caption{Performance as a function of number of features used for
%   S\"{u}mb\"{u}l dataset. For details about the box plot, see Figure~\ref{fig:exhaustive}.}
% \end{center}
% \label{fig:sumbulexhaustive}
% \end{figure}


% \begin{figure}[!ht]
% \begin{center}
% \includegraphics{FIGS/Sumbul/KnownSumbul-Confidence-in-classification-leave-one-out.eps}
% \end{center}
% \caption{Confidence of classifier in prediction. For details see Figure~\ref{fig:confidence}.}
% \label{fig:sumbulconfidence}
% \end{figure}

% \begin{figure}[!ht]
% \begin{center}
% \includegraphics{FIGS/Sumbul/PlotSpace-for-article-summary-knownSumbul.eps}
% \end{center}
% \caption{Feature space for the four selected features for the \Sumbul
%   dataset.}
% \label{fig:sumbulfeaturespace}
% \end{figure}

% \begin{figure}[!ht]
% \begin{center}
% \includegraphics{FIGS/Sumbul/SVM-NaiveBayes-comparison-knownSumbul-DATA.eps}
% \caption{Performance of SVM, Naive Bayes and BAGS for the best
%   N-feature sets found during the exhaustive search (using Naive Bayes).}
% \label{fig:sumbulmethodcomparison}
% \end{center}
% \end{figure}



% \begin{sidewaystable}
% \begin{tabular}{lrrrrrrrrrrrrrr}
% & BD& BA& DA& DD& DDi& DBP& FDBC& MBA& MSL& MST& MTSL& NBP& SD& TDL\\
% \hline
% Bistratification Distance (BD) & \textbf{}  &  &  &  &  &  &  &  &  &  &  &  &  & \\
% Branch Assymetry (BA) & 0.25 & \textbf{}  &  &  &  &  &  &  &  &  &  &  &  & \\
% Dendritic Area (DA) & -0.14 & \textbf{-0.44} & \textbf{}  &  &  &  &  &  &  &  &  &  &  & \\
% Dendritic Density (DD) & 0.07 & \textbf{0.42} & \textbf{-0.58} & \textbf{}  &  &  &  &  &  &  &  &  &  & \\
% Dendritic Diameter (DDi) & -0.12 & \textbf{-0.41} & \textbf{0.94} & \textbf{-0.67} & \textbf{}  &  &  &  &  &  &  &  &  & \\
% Density of Branch Points (DBP) & 0.04 & \textbf{0.45} & \textbf{-0.50} & \textbf{0.95} & \textbf{-0.59} & \textbf{}  &  &  &  &  &  &  &  & \\
% Fractal Dimension Box Counting (FDBC) & 0.15 & \textbf{0.41} & \textbf{-0.49} & \textbf{0.86} & \textbf{-0.54} & \textbf{0.74} & \textbf{}  &  &  &  &  &  &  & \\
% Mean Branch Angle (MBA) & 0.10 & 0.24 & -0.28 & 0.21 & \textbf{-0.30} & 0.22 & 0.20 & \textbf{}  &  &  &  &  &  & \\
% Mean Segment Length (MSL) & -0.26 & \textbf{-0.64} & \textbf{0.78} & \textbf{-0.58} & \textbf{0.75} & \textbf{-0.54} & \textbf{-0.61} & \textbf{-0.36} & \textbf{}  &  &  &  &  & \\
% Mean Segment Tortuosity (MST) & -0.24 & -0.26 & 0.06 & -0.18 & 0.05 & -0.23 & -0.29 & -0.01 & \textbf{0.36} & \textbf{}  &  &  &  & \\
% Mean Terminal Segment Length (MTSL) & -0.27 & \textbf{-0.64} & \textbf{0.75} & \textbf{-0.52} & \textbf{0.71} & \textbf{-0.49} & \textbf{-0.55} & \textbf{-0.36} & \textbf{0.99} & \textbf{0.34} & \textbf{}  &  &  & \\
% Number of Branch Points (NBP) & 0.23 & \textbf{0.59} & \textbf{-0.31} & \textbf{0.64} & \textbf{-0.33} & \textbf{0.67} & \textbf{0.73} & 0.22 & \textbf{-0.68} & \textbf{-0.51} & \textbf{-0.64} & \textbf{}  &  & \\
% Stratification Depth (SD) & 0.18 & 0.24 & -0.12 & 0.15 & -0.09 & 0.08 & 0.19 & 0.05 & -0.07 & -0.09 & -0.08 & 0.11 & \textbf{}  & \\
% Total Dendritic Length (TDL) & 0.03 & -0.15 & \textbf{0.67} & -0.15 & \textbf{0.66} & -0.18 & 0.17 & -0.17 & \textbf{0.32} & -0.19 & \textbf{0.31} & 0.25 & 0.10 & \textbf{} \\
% \hline
% \end{tabular}
% \caption{\Sumbul dataset, including only known genetically labelled 
%   cells. Using P threshold 0.30}
% \label{tab:sumbulcorr}\end{sidewaystable}

% runExhaustiveFeatureSearchSumbul    % !!! This is slow
% analyseExhaustiveFeatureSearchSumbul
% 


\begin{sidewaystable}
\begin{tabular}{ccllllllllllllll}
Number of features & Performance  & BD & BA & DA & DD & DDi & DBP & FD & MBA & MSL & MST & MTSL & NBP & SD & TDL\\
\hline
1 & $56.0 \pm 1.4\,\%$  &  &  &  &  &  &  & $\bullet$ &  &  &  &  &  &  & \\
2 & $67.1 \pm 2.0\,\%$  &  &  &  &  &  &  & $\bullet$ &  &  &  &  &  & $\bullet$ & \\
3 & $75.8 \pm 2.2\,\%$  &  &  &  & $\bullet$ &  &  &  &  &  &  &  &  & $\bullet$ & $\bullet$\\
4 & $81.1 \pm 2.0\,\%$  & $\bullet$ &  &  &  &  & $\bullet$ &  &  &  &  &  &  & $\bullet$ & $\bullet$\\
5 & $81.6 \pm 2.7\,\%$  & $\bullet$ &  &  &  &  &  & $\bullet$ &  &  &  & $\bullet$ &  & $\bullet$ & $\bullet$\\
6 & $81.9 \pm 2.4\,\%$  & $\bullet$ &  &  &  & $\bullet$ &  & $\bullet$ &  &  &  & $\bullet$ &  & $\bullet$ & $\bullet$\\
7 & $80.4 \pm 2.2\,\%$  & $\bullet$ &  &  &  & $\bullet$ &  & $\bullet$ &  &  & $\bullet$ & $\bullet$ &  & $\bullet$ & $\bullet$\\
8 & $79.8 \pm 2.3\,\%$  & $\bullet$ & $\bullet$ &  &  & $\bullet$ &  & $\bullet$ & $\bullet$ &  & $\bullet$ &  &  & $\bullet$ & $\bullet$\\
9 & $79.0 \pm 1.9\,\%$  & $\bullet$ & $\bullet$ &  &  & $\bullet$ &  & $\bullet$ & $\bullet$ &  & $\bullet$ & $\bullet$ &  & $\bullet$ & $\bullet$\\
10 & $78.1 \pm 1.8\,\%$  & $\bullet$ & $\bullet$ &  &  & $\bullet$ &  & $\bullet$ & $\bullet$ &  & $\bullet$ & $\bullet$ & $\bullet$ & $\bullet$ & $\bullet$\\
11 & $76.6 \pm 2.0\,\%$  & $\bullet$ & $\bullet$ &  &  & $\bullet$ &  & $\bullet$ & $\bullet$ & $\bullet$ & $\bullet$ & $\bullet$ & $\bullet$ & $\bullet$ & $\bullet$\\
12 & $74.7 \pm 2.0\,\%$  & $\bullet$ & $\bullet$ & $\bullet$ &  & $\bullet$ &  & $\bullet$ & $\bullet$ & $\bullet$ & $\bullet$ & $\bullet$ & $\bullet$ & $\bullet$ & $\bullet$\\
13 & $73.0 \pm 2.3\,\%$  & $\bullet$ & $\bullet$ & $\bullet$ & $\bullet$ & $\bullet$ & $\bullet$ & $\bullet$ & $\bullet$ &  & $\bullet$ & $\bullet$ & $\bullet$ & $\bullet$ & $\bullet$\\
14 & $71.6 \pm 2.1\,\%$  & $\bullet$ & $\bullet$ & $\bullet$ & $\bullet$ & $\bullet$ & $\bullet$ & $\bullet$ & $\bullet$ & $\bullet$ & $\bullet$ & $\bullet$ & $\bullet$ & $\bullet$ & $\bullet$\\
\bottomrule
\end{tabular}
\caption{Feature set performance for \Sumbul dataset. See Table
  \ref{tab:performance} for abbreviations.}
\label{tab:sumbulfeatureselection}
\end{sidewaystable}




% \begin{sidewaystable}
% \begin{tabular}{llllllll}
%  & \multicolumn{5}{c}{Cell Type}\\
% \cline{2-6}
% Feature name & CB2 & Cdh3 & BD & JAM-B & K & W3 & W7\\
% \hline
% Bistratification Distance ($\mu m$)& $1.5 \pm 0.7$& $1.8 \pm 1.4$& $5.5 \pm 2.9$& $2.6 \pm 1.5$& $1.4 \pm 1.3$& $1.2 \pm 1.2$& $0.7 \pm 0.7$\\
% Branch Assymetry& $0.67 \pm 0.08$& $0.62 \pm 0.03$& $0.69 \pm 0.02$& $0.68 \pm 0.04$& $0.64 \pm 0.02$& $0.69 \pm 0.02$& $0.63 \pm 0.02$\\
% Dendritic Area ($mm^2$)& $0.03 \pm 0.01$& $0.04 \pm 0.01$& $0.04 \pm 0.02$& $0.03 \pm 0.01$& $0.06 \pm 0.02$& $0.01 \pm 0.00$& $0.06 \pm 0.01$\\
% Dendritic Density ($\mu m^{-1}$)& $0.11 \pm 0.02$& $0.06 \pm 0.01$& $0.14 \pm 0.06$& $0.11 \pm 0.02$& $0.08 \pm 0.02$& $0.34 \pm 0.09$& $0.08 \pm 0.01$\\
% Dendritic Diameter ($\mu m$)& $257 \pm 43$& $272 \pm 51$& $261 \pm 68$& $243 \pm 36$& $310 \pm 52$& $139 \pm 17$& $336 \pm 45$\\
% Density of Branch Points ($mm^{-2}$)& $2241 \pm 1281$& $758 \pm 346$& $5515 \pm 4404$& $2790 \pm 1341$& $1340 \pm 815$& $19799 \pm 7003$& $1117 \pm 447$\\
% Fractal Dimension Box Counting& $1.49 \pm 0.05$& $1.35 \pm 0.04$& $1.54 \pm 0.04$& $1.47 \pm 0.04$& $1.45 \pm 0.04$& $1.66 \pm 0.08$& $1.46 \pm 0.03$\\
% Mean Branch Angle (degrees)& $84 \pm 9$& $85 \pm 12$& $89 \pm 4$& $86 \pm 8$& $79 \pm 6$& $92 \pm 4$& $78 \pm 8$\\
% Mean Segment Length ($\mu m$)& $30.2 \pm 11.1$& $42.1 \pm 9.9$& $16.5 \pm 5.8$& $23.7 \pm 7.5$& $39.1 \pm 18.1$& $9.4 \pm 2.9$& $39.5 \pm 12.9$\\
% Mean Segment Tortuosity& $1.24 \pm 0.04$& $1.30 \pm 0.03$& $1.20 \pm 0.03$& $1.25 \pm 0.04$& $1.25 \pm 0.04$& $1.24 \pm 0.07$& $1.22 \pm 0.04$\\
% Mean Terminal Segment Length ($\mu m$)& $37.6 \pm 18.4$& $54.8 \pm 18.1$& $16.5 \pm 6.7$& $22.6 \pm 9.1$& $55.1 \pm 32.3$& $10.7 \pm 3.5$& $51.8 \pm 21.2$\\
% Number of Branch Points& $69 \pm 34$& $25 \pm 9$& $154 \pm 40$& $81 \pm 36$& $60 \pm 16$& $196 \pm 63$& $69 \pm 25$\\
% Stratification Depth ($\mu m$)& $23 \pm 12$& $6 \pm 4$& $28 \pm 5$& $37 \pm 7$& $9 \pm 8$& $25 \pm 5$& $27 \pm 9$\\
% Total Dendritic Length ($mm$)& $3.5 \pm 0.4$& $2.1 \pm 0.5$& $4.8 \pm 1.4$& $3.3 \pm 0.6$& $4.3 \pm 0.9$& $3.4 \pm 0.9$& $4.9 \pm 0.8$\\
% \hline
% \end{tabular}
% \caption{Average feature values for genetic classes in \Sumbul
%   dataset.}
% \label{tab:sumbulclasscharacteristics}
% \end{sidewaystable}



% \begin{table}
% \begin{tabular}{lrrrrrrrr}
%  & 1 & 2 & 3 & 4 & 5 & 6 & 7 & 8\\
% \hline
% CB2 & 2 & 1 & 0 & 1 & 0 & 5 & 0 & 0\\
% Cdh3 & 0 & 0 & 0 & 0 & 10 & 0 & 0 & 0\\
% BD & 0 & 4 & 2 & 0 & 0 & 0 & 17 & 0\\
% JAM-B & 0 & 0 & 0 & 31 & 0 & 1 & 0 & 0\\
% K & 0 & 0 & 0 & 0 & 0 & 9 & 0 & 3\\
% W3 & 0 & 15 & 0 & 0 & 1 & 0 & 0 & 0\\
% W7 & 3 & 0 & 4 & 0 & 0 & 2 & 0 & 0\\
% \hline
% \end{tabular}
% \caption{Blind clustering of the Sumbul data. All cells were included
%   in the optimization, the rand index for the genetically labeled
%   cells were calculated. Clustering with k-means, k=7,...,20. Here k=8
% gave best results using the 8 features: biStratificationDistance branchAssymetry fractalDimensionBoxCounting meanSegmentTortuosity meanTerminalSegmentLength numBranchPoints stratificationDepth totalDendriticLength. }
% \end{table}


\bibliography{RefLibrary}


\end{document}
