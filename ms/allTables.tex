\documentclass[11pt]{article}

\usepackage{amsmath}
\usepackage{amssymb}

\usepackage{mathpazo}

\usepackage{graphicx}

\usepackage{cite}

\usepackage{color} 

% Use doublespacing - comment out for single spacing
\usepackage{setspace} 
\onehalfspacing
%\doublespacing

\usepackage[utf8]{inputenc} 

\usepackage{dcolumn}
\newcolumntype{.}{D{.}{.}{-1}}

\usepackage{marginnote}
%%\newcommand{\mycomment}[1]{\marginnote{#1}}
\newcommand{\mycomment}[1]{}
% use \mycomment{comment text}

% Boolean expressions
% \usepackage{etoolbox}
%
% \newtoggle{groupResults}
% \toggletrue{groupResults}
% \togglefalse{groupResults}


\usepackage[a4paper,margin=3cm]{geometry}
\usepackage{url}
% Bold the 'Figure #' in the caption and separate it with a period
% Captions will be left justified
\usepackage[labelfont=bf,labelsep=period,justification=raggedright]{caption}

% Get formating for table
\usepackage{siunitx}

\usepackage{booktabs}

\usepackage{lastpage}

% Rotated table
\usepackage{rotating}


% J Neurosci stylesheet - not working!
\usepackage{namedplus}
\bibliographystyle{namedplus}
% Use the PLoS provided bibtex style
%\bibliographystyle{plos2009}
% \bibliographystyle{modelcomp}

% Remove brackets from numbering in List of References
% \makeatletter
% \renewcommand{\@biblabel}[1]{\quad#1.}
% \makeatother

% Space after your macros, latex being obnoxious
\usepackage{xspace}

\usepackage{multirow}

% Turn of hyphenation to make word count more accurate
% \usepackage[none]{hyphenat} 
\tolerance=1
\emergencystretch=\maxdimen
\hyphenpenalty=10000
\hbadness=10000
\sloppy

% Leave date blank
\date{}

\pagestyle{myheadings}
%% ** EDIT HERE **

\newcommand{\Sumbul}{S\"{u}mb\"{u}l\xspace}


%% ** EDIT HERE **
%% PLEASE INCLUDE ALL MACROS BELOW

\DeclareMathOperator*{\argmin}{arg\,min}

% pedex and apex, subscript and superscript

\providecommand*{\ped}[1]{%
\ensuremath{_\mathrm{#1}}}
\providecommand*{\ap}[1]{%
\ensuremath{^\mathrm{#1}}}
\newcommand{\gMathV}{\textit{Math5}$\ap{\mathrm{-/-}}$\xspace}
\newcommand{\pMathV}{Math5$\ap{\mathrm{-/-}}$\xspace}

\newcommand{\gIsl}{\textit{Isl2}\xspace}
\newcommand{\gIslE}{\textit{Isl2-EphA3}\xspace}
\newcommand{\gIslEkk}{\textit{Isl2-EphA3}$\ap{\mathrm{ki/ki}}$\xspace}
\newcommand{\gIslEkp}{\textit{Isl2-EphA3}$\ap{\mathrm{ki/+}}$\xspace}

\newcommand{\gIslp}{\textrm{Isl2\ap{+}}\xspace}
\newcommand{\gIslm}{\textrm{Isl2\ap{-}}\xspace}

\newcommand{\gTKO}{\textit{{ephrin-A2,A3,A5}}\xspace}
\newcommand{\gephrinA}{\textit{ephrin-A}\xspace}
\newcommand{\gEphA}{\textit{EphA}\xspace}

\newcommand{\gEphAIII}{\textit{EphA3}\xspace}
\newcommand{\gEphAIV}{\textit{EphA4}\xspace}
\newcommand{\gEphAV}{\textit{EphA5}\xspace}

\usepackage{upgreek}
%\newcommand{\betaIIKO}{\ensuremath{\upbeta 2^{-/-}}\xspace}
\newcommand{\betaIIKO}{\ensuremath{\beta\mathit{2}^{-/-}}\xspace}

%% END MACROS SECTION

\begin{document}

\singlespacing

\doublespacing

\newcommand{\etal}{et al.\ }
\begin{table}
\centering
\begin{tabular}{p{4.5cm} p{4.5cm} lr}
  \toprule
  Genetic type & Description & Citation & N\\
  \midrule
  \footnotesize{Calretinin (CB2)} & \footnotesize{Transient Off-alpha}
  & \footnotesize{Huberman \etal (2008)}& 19\\
  \footnotesize{Cadherin-3 (Cdh3)} & \footnotesize{M2 ipRGCs \&
    "diving cells"} & \footnotesize{Osterhout \etal (2011)}&11\\
  \footnotesize{Dopamine Receptor 4 (DRD4)} & \footnotesize{On-Off
    DS (posterior motion)} & \footnotesize{Huberman
    \etal (2009)}&26\\
  \footnotesize{Homeobox d10 (Hoxd10)} & \footnotesize{On-DS (3 types)
    and On-Off DS/anterior tuned (1 type)} 
  & \footnotesize{Dhande \etal (2013)} & 29\\
  \footnotesize{Thyrotropin-Releasing Hormone Receptor (TRHR)} &
  \footnotesize{On-Off DS (posterior motion)}&
  \footnotesize{Rivlin Etzion \etal (2011)} & 9\\
  \bottomrule
\end{tabular}
\caption{Five genetic types in this study.  N is
  the number of cells analysed.}
\label{tab:geneticTypes}
\end{table}


\clearpage

\singlespacing

\begin{table}
  \centering
  {\renewcommand{\arraystretch}{1.3} \begin{tabular}{lp{8.5cm}}
      \toprule
      \textbf{Feature} & \textbf{Description} \\
      \midrule
      BD: Bistratification Distance & {\footnotesize Distance between putative bistratified dendrites (normalised by variance)}\\
      BA: Branch Asymmetry & {\footnotesize Ratio of number of leaves
        on either side of a branch point}\\
      DA: Dendritic Area & {\footnotesize Area of dendritic arbor projected on the XY-plane, calculated using the convex hull}\\
      DD: Dendritic Density & {\footnotesize TDL / DA}\\
      DDi: Dendritic Diameter & {\footnotesize  Largest distance between two points on the dendritic arbor}\\
      DBP: Density of Branch Points &  {\footnotesize  NBP / DA }\\
      FD: Fractal Dimension & {\footnotesize  Fractal-based dendritic complexity measure}\\
      MBA: Mean Branch Angle &  {\footnotesize  Mean angle between the branches in 3D space.}\\
      MSL: Mean Segment Length & {\footnotesize  Mean length of dendritic segments}\\
      MST: Mean Segment Tortuosity & {\footnotesize  Ratio of segment length to distance between end points}\\
      MTSL: Mean Terminal Segment Length & {\footnotesize  Mean length of the last segment of each dendrite}\\
      NBP: Number of Branch Points &  {\footnotesize  Number of points where the dendrites branch}\\
      SA: Soma Area & {\footnotesize  Area of the soma projected onto the XY-plane}\\
      SD: Stratification Depth & {\footnotesize Centre of mass of dendrite along z-axis, with soma at $z=0$}\\
      TDL: Total Dendritic Length & {\footnotesize Total length of all dendrites}\\
  \bottomrule
\end{tabular}
}
\caption{Features calculated from each RGC.}
\label{tab:featurelist}
\end{table}




\clearpage

%%%%%%%%%%%%%%%%%%%%%%%%%%%%%%%%%%%%%%%%%%%%%%%%%%%%%%%%%%%%%%%%%%%%%%%%%%%%
%
% Generated by featureValueTable.m
%

\begin{sidewaystable}
\begin{tabular}{llllll}
\toprule
 & \multicolumn{5}{c}{Genetic type}\\
% \cline{2-6}
Feature name & CB2 & Cdh3 & DRD4 & Hoxd10 & TRHR\\
\midrule
Bistratification Distance ($\mu m$)& $1.5 \pm 1.1$& $3.7 \pm 4.1$& $2.3 \pm 1.1$& $1.9 \pm 2.1$& $3.2 \pm 2.1$\\
Branch Asymmetry& $0.63 \pm 0.02$& $0.74 \pm 0.02$& $0.71 \pm 0.03$& $0.67 \pm 0.03$& $0.71 \pm 0.03$\\
Dendritic Area ($mm^2$)& $0.06 \pm 0.02$& $0.04 \pm 0.01$& $0.05 \pm 0.01$& $0.07 \pm 0.04$& $0.03 \pm 0.01$\\
Dendritic Density ($\mu m^{-1}$)& $0.07 \pm 0.01$& $0.12 \pm 0.03$& $0.13 \pm 0.02$& $0.08 \pm 0.03$& $0.16 \pm 0.02$\\
Dendritic Diameter ($\mu m$)& $326 \pm 75$& $270 \pm 52$& $300 \pm 37$& $358 \pm 104$& $248 \pm 34$\\
Density of Branch Points ($mm^{-2}$)& $1351 \pm 675$& $7697 \pm 2758$& $5235 \pm 1031$& $2295 \pm 1785$& $7150 \pm 1466$\\
Fractal Dimension& $1.40 \pm 0.02$& $1.47 \pm 0.04$& $1.49 \pm 0.03$& $1.44 \pm 0.06$& $1.54 \pm 0.04$\\
Mean Branch Angle ($^\circ$)& $100 \pm 4$& $99 \pm 5$& $104 \pm 3$& $103 \pm 4$& $102 \pm 3$\\
Mean Segment Length ($\mu m$)& $29.5 \pm 7.1$& $7.9 \pm 1.5$& $12.7 \pm 2.4$& $22.3 \pm 7.9$& $11.2 \pm 2.2$\\
Mean Segment Tortuosity& $1.15 \pm 0.02$& $1.21 \pm 0.05$& $1.20 \pm 0.05$& $1.20 \pm 0.07$& $1.17 \pm 0.03$\\
Mean Terminal Segment Length ($\mu m$)& $36.3 \pm 10.1$& $6.3 \pm 1.2$& $11.8 \pm 2.7$& $20.9 \pm 8.9$& $10.0 \pm 3.0$\\
Number of Branch Points& $70 \pm 12$& $280 \pm 87$& $242 \pm 66$& $124 \pm 47$& $233 \pm 69$\\
Soma Area ($\mu m^2$)& $348 \pm 92$& $141 \pm 34$& $180 \pm 52$& $202 \pm 74$& $190 \pm 51$\\
Stratification Depth ($\mu m$)& $-7 \pm 3$& $-10 \pm 3$& $-11 \pm 4$& $-11 \pm 4$& $-10 \pm 4$\\
Total Dendritic Length ($mm$)& $4.3 \pm 1.2$& $4.4 \pm 1.0$& $6.0 \pm 1.1$& $5.2 \pm 1.6$& $5.1 \pm 0.9$\\
\bottomrule
\end{tabular}
\caption{Mean and standard deviation of features for each genetic type.}
\label{tab:featVals}
\end{sidewaystable}


\clearpage

%%%%%%%%%%%%%%%%%%%%%%%%%%%%%%%%%%%%%%%%%%%%%%%%%%%%%%%%%%%%%%%%%%%%%%%%%%%%

% r = RGCclass(0);
% r.lazyLoad();
% r.featureCorrelation()


\begin{sidewaystable}
\begin{tabular}{lrrrrrrrrrrrrrrr}
\toprule
& BD& BA& DA& DD& DDi& DBP& FD& MBA& MSL& MST& MTSL& NBP& SA& SD& TDL\\
\midrule
Bistratification Distance (BD) & \textbf{}  &  &  &  &  &  &  &  &  &  &  &  &  &  & \\
Branch Asymmetry (BA) & 0.25 & \textbf{}  &  &  &  &  &  &  &  &  &  &  &  &  & \\
Dendritic Area (DA) & -0.37 & -0.27 & \textbf{}  &  &  &  &  &  &  &  &  &  &  &  & \\
Dendritic Density (DD) & \textbf{0.47} & \textbf{0.52} & \textbf{-0.74} & \textbf{}  &  &  &  &  &  &  &  &  &  &  & \\
Dendritic Diameter (DDi) & -0.37 & -0.20 & \textbf{0.96} & \textbf{-0.69} & \textbf{}  &  &  &  &  &  &  &  &  &  & \\
Density of Branch Points (DBP) & \textbf{0.51} & \textbf{0.79} & \textbf{-0.63} & \textbf{0.83} & \textbf{-0.60} & \textbf{}  &  &  &  &  &  &  &  &  & \\
Fractal Dimension (FD) & 0.38 & \textbf{0.50} & \textbf{-0.52} & \textbf{0.84} & \textbf{-0.47} & \textbf{0.70} & \textbf{}  &  &  &  &  &  &  &  & \\
Mean Branch Angle (MBA) & 0.13 & -0.02 & 0.02 & 0.13 & 0.06 & -0.07 & -0.00 & \textbf{}  &  &  &  &  &  &  & \\
Mean Segment Length (MSL) & -0.36 & \textbf{-0.81} & \textbf{0.65} & \textbf{-0.73} & \textbf{0.60} & \textbf{-0.85} & \textbf{-0.64} & -0.01 & \textbf{}  &  &  &  &  &  & \\
Mean Segment Tortuosity (MST) & 0.16 & 0.10 & -0.06 & 0.24 & -0.03 & 0.11 & 0.05 & 0.29 & -0.07 & \textbf{}  &  &  &  &  & \\
Mean Terminal Segment Length (MTSL) & -0.31 & \textbf{-0.80} & \textbf{0.52} & \textbf{-0.65} & \textbf{0.49} & \textbf{-0.78} & \textbf{-0.59} & -0.05 & \textbf{0.97} & -0.13 & \textbf{}  &  &  &  & \\
Number of Branch Points (NBP) & 0.32 & \textbf{0.90} & -0.25 & \textbf{0.60} & -0.18 & \textbf{0.81} & \textbf{0.60} & -0.03 & \textbf{-0.78} & 0.09 & \textbf{-0.75} & \textbf{}  &  &  & \\
Soma Area (SA) & -0.15 & \textbf{-0.44} & 0.21 & -0.32 & 0.19 & -0.39 & -0.21 & -0.26 & \textbf{0.44} & \textbf{-0.47} & \textbf{0.52} & -0.35 & \textbf{}  &  & \\
Stratification Depth (SD) & -0.13 & -0.04 & 0.28 & -0.32 & 0.27 & -0.13 & -0.24 & -0.10 & 0.23 & \textbf{-0.44} & 0.25 & 0.04 & 0.22 & \textbf{}  & \\
Total Dendritic Length (TDL) & -0.09 & 0.27 & \textbf{0.58} & 0.02 & \textbf{0.63} & -0.02 & 0.21 & 0.18 & 0.06 & 0.15 & -0.01 & \textbf{0.45} & -0.02 & 0.12 & \textbf{} \\
\bottomrule
\end{tabular}
\caption{Correlation coefficient calculated for pairs of 15
  features. Features which are highly correlated or anti-correlated
  are marked in bold.}
\label{tab:corr}\end{sidewaystable}




% blindClusteringBatch.m

% SAVES files in RESULTS
% from: RESULTS/BlindClustering-latex-n-5.tex

% This table uses our 5-feature set
\begin{table}
\centering
\begin{tabular}{llrrrrr}
\toprule
 & & \multicolumn{5}{c}{Cluster}\\
 & & 1 & 2 & 3 & 4 & 5\\
%%\cline{3-7}
\midrule
\multirow{5}{*}{\rotatebox{90}{Genetic type}}& CB2 & 0 & 11 & 0 & 0 & 8\\
& Cdh3 & 0 & 0 & 4 & 7 & 0\\
& DRD4 & 0 & 0 & 20 & 6 & 0\\
& Hoxd10 & 12 & 2 & 12 & 2 & 1\\
& TRHR & 0 & 0 & 0 & 9 & 0\\
\bottomrule
\end{tabular}
\caption{Comparison of blind clustering with genetic type.}
\label{tab:blind5confusion}
\end{table}



\clearpage


% runExhaustiveFeatureSearch     % !!! This is slow
% analyseExhaustiveFeatureSearch
% 

% 20 reps
\begin{sidewaystable}
\begin{tabular}{cclllllllllllllll}
Number of features & Performance  & BD & BA & DA & DD & DDi & DBP & FD & MBA & MSL & MST & MTSL & NBP & SA & SD & TDL\\
\hline
1 & $64.7 \pm 1.7\,\%$  &  &  &  &  &  &  &  &  &  &  & $\bullet$ &  &  &  & \\
2 & $72.9 \pm 2.0\,\%$  &  &  &  &  &  & $\bullet$ &  &  &  &  &  &  & $\bullet$ &  & \\
3 & $78.7 \pm 2.8\,\%$  &  &  &  &  &  &  & $\bullet$ &  &  &  & $\bullet$ &  & $\bullet$ &  & \\
4 & $80.7 \pm 3.1\,\%$  &  &  & $\bullet$ &  &  &  & $\bullet$ &  &  &  & $\bullet$ &  & $\bullet$ &  & \\
5 & $83.1 \pm 3.6\,\%$  &  &  & $\bullet$ &  &  & $\bullet$ & $\bullet$ &  &  &  & $\bullet$ &  & $\bullet$ &  & \\
6 & $84.0 \pm 2.7\,\%$  &  &  & $\bullet$ &  &  &  & $\bullet$ &  & $\bullet$ & $\bullet$ &  & $\bullet$ & $\bullet$ &  & \\
7 & $85.5 \pm 2.9\,\%$  &  &  & $\bullet$ &  &  & $\bullet$ & $\bullet$ &  &  & $\bullet$ & $\bullet$ & $\bullet$ & $\bullet$ &  & \\
8 & $86.3 \pm 2.6\,\%$  &  &  & $\bullet$ &  &  & $\bullet$ & $\bullet$ & $\bullet$ &  & $\bullet$ & $\bullet$ & $\bullet$ & $\bullet$ &  & \\
9 & $85.7 \pm 2.1\,\%$  & $\bullet$ &  & $\bullet$ & $\bullet$ &  &  & $\bullet$ &  &  & $\bullet$ & $\bullet$ & $\bullet$ & $\bullet$ &  & $\bullet$\\
10 & $85.1 \pm 2.2\,\%$  &  &  & $\bullet$ &  &  & $\bullet$ & $\bullet$ & $\bullet$ & $\bullet$ & $\bullet$ & $\bullet$ & $\bullet$ & $\bullet$ &  & $\bullet$\\
11 & $85.2 \pm 1.7\,\%$  &  &  & $\bullet$ & $\bullet$ & $\bullet$ & $\bullet$ & $\bullet$ &  & $\bullet$ & $\bullet$ & $\bullet$ & $\bullet$ & $\bullet$ &  & $\bullet$\\
12 & $84.9 \pm 1.7\,\%$  &  &  & $\bullet$ & $\bullet$ & $\bullet$ & $\bullet$ & $\bullet$ & $\bullet$ & $\bullet$ & $\bullet$ & $\bullet$ & $\bullet$ & $\bullet$ &  & $\bullet$\\
13 & $84.1 \pm 1.7\,\%$  &  & $\bullet$ & $\bullet$ & $\bullet$ & $\bullet$ & $\bullet$ & $\bullet$ & $\bullet$ & $\bullet$ & $\bullet$ & $\bullet$ & $\bullet$ & $\bullet$ &  & $\bullet$\\
14 & $83.2 \pm 2.0\,\%$  & $\bullet$ & $\bullet$ & $\bullet$ & $\bullet$ & $\bullet$ & $\bullet$ & $\bullet$ & $\bullet$ & $\bullet$ & $\bullet$ & $\bullet$ & $\bullet$ & $\bullet$ &  & $\bullet$\\
15 & $82.2 \pm 1.8\,\%$  & $\bullet$ & $\bullet$ & $\bullet$ & $\bullet$ & $\bullet$ & $\bullet$ & $\bullet$ & $\bullet$ & $\bullet$ & $\bullet$ & $\bullet$ & $\bullet$ & $\bullet$ & $\bullet$ & $\bullet$\\
\bottomrule
\end{tabular}
\caption{Performance of classifiers. Five-fold cross-validation,
  repeated twenty times for each feature set. Bistratification
  Distance (BD), Branch Asymmetry (BA), Dendritic Area (DA), Dendritic
  Density (DD), Dendritic Diameter (DDi), Density of Branch Points
  (DBP), Fractal Dimension (FD), Mean Branch Angle
  (MBA), Mean Segment Length (MSL), Mean Segment Tortuosity (MST),
  Mean Terminal Segment Length (MTSL), Number of Branch Points (NBP),
  Soma Area (SA), Stratification Depth (SD), Total Dendritic Length
  (TDL). Performance is given as mean $\pm$ standard deviation
    correctly classified.}
\label{tab:performance}
\end{sidewaystable}



\clearpage

% getConfusionMatrixLatex

% !!! OBS, need to make diagonal elements bold by hand! Since the general latex generating function I wrote
% does not allow for that, it would affect other tables alos.

\begin{table}
\centering
\begin{tabular}{llrrrrr}
\toprule
 & & \multicolumn{5}{c}{Predicted type}\\
\cline{3-7}
 & & CB2 & Cdh3 & DRD4 & Hoxd10 & TRHR\\
\midrule
\multirow{5}{*}{\rotatebox{90}{Genetic type}}& CB2 & \textbf{18} & 0 & 0 & 1 & 0\\
& Cdh3 & 0 & \textbf{9} & 0 & 1 & 1\\
& DRD4 & 0 & 0 & \textbf{25} & 0 & 1\\
& Hoxd10 & 2 & 1 & 1 & \textbf{23} & 2\\
& TRHR & 0 & 1 & 3 & 0 & \textbf{5}\\
\bottomrule % \cline{3-7}
\end{tabular}
\caption{Confusion matrix **result**.}
\label{tab:confusionMatrixLeaveOneOut}
\end{table}

\clearpage

% runExhaustiveFeatureSearchSumbul    % !!! This is slow
% analyseExhaustiveFeatureSearchSumbul
% 


\begin{sidewaystable}
\begin{tabular}{ccllllllllllllll}
Number of features & Performance  & BD & BA & DA & DD & DDi & DBP & FD & MBA & MSL & MST & MTSL & NBP & SD & TDL\\
\hline
1 & $56.0 \pm 1.4\,\%$  &  &  &  &  &  &  & $\bullet$ &  &  &  &  &  &  & \\
2 & $67.1 \pm 2.0\,\%$  &  &  &  &  &  &  & $\bullet$ &  &  &  &  &  & $\bullet$ & \\
3 & $75.8 \pm 2.2\,\%$  &  &  &  & $\bullet$ &  &  &  &  &  &  &  &  & $\bullet$ & $\bullet$\\
4 & $81.1 \pm 2.0\,\%$  & $\bullet$ &  &  &  &  & $\bullet$ &  &  &  &  &  &  & $\bullet$ & $\bullet$\\
5 & $81.6 \pm 2.7\,\%$  & $\bullet$ &  &  &  &  &  & $\bullet$ &  &  &  & $\bullet$ &  & $\bullet$ & $\bullet$\\
6 & $81.9 \pm 2.4\,\%$  & $\bullet$ &  &  &  & $\bullet$ &  & $\bullet$ &  &  &  & $\bullet$ &  & $\bullet$ & $\bullet$\\
7 & $80.4 \pm 2.2\,\%$  & $\bullet$ &  &  &  & $\bullet$ &  & $\bullet$ &  &  & $\bullet$ & $\bullet$ &  & $\bullet$ & $\bullet$\\
8 & $79.8 \pm 2.3\,\%$  & $\bullet$ & $\bullet$ &  &  & $\bullet$ &  & $\bullet$ & $\bullet$ &  & $\bullet$ &  &  & $\bullet$ & $\bullet$\\
9 & $79.0 \pm 1.9\,\%$  & $\bullet$ & $\bullet$ &  &  & $\bullet$ &  & $\bullet$ & $\bullet$ &  & $\bullet$ & $\bullet$ &  & $\bullet$ & $\bullet$\\
10 & $78.1 \pm 1.8\,\%$  & $\bullet$ & $\bullet$ &  &  & $\bullet$ &  & $\bullet$ & $\bullet$ &  & $\bullet$ & $\bullet$ & $\bullet$ & $\bullet$ & $\bullet$\\
11 & $76.6 \pm 2.0\,\%$  & $\bullet$ & $\bullet$ &  &  & $\bullet$ &  & $\bullet$ & $\bullet$ & $\bullet$ & $\bullet$ & $\bullet$ & $\bullet$ & $\bullet$ & $\bullet$\\
12 & $74.7 \pm 2.0\,\%$  & $\bullet$ & $\bullet$ & $\bullet$ &  & $\bullet$ &  & $\bullet$ & $\bullet$ & $\bullet$ & $\bullet$ & $\bullet$ & $\bullet$ & $\bullet$ & $\bullet$\\
13 & $73.0 \pm 2.3\,\%$  & $\bullet$ & $\bullet$ & $\bullet$ & $\bullet$ & $\bullet$ & $\bullet$ & $\bullet$ & $\bullet$ &  & $\bullet$ & $\bullet$ & $\bullet$ & $\bullet$ & $\bullet$\\
14 & $71.6 \pm 2.1\,\%$  & $\bullet$ & $\bullet$ & $\bullet$ & $\bullet$ & $\bullet$ & $\bullet$ & $\bullet$ & $\bullet$ & $\bullet$ & $\bullet$ & $\bullet$ & $\bullet$ & $\bullet$ & $\bullet$\\
\bottomrule
\end{tabular}
\caption{Feature set performance for \Sumbul dataset. See Table
  \ref{tab:performance} for abbreviations.}
\label{tab:sumbulfeatureselection}
\end{sidewaystable}



\bibliography{RefLibrary}


\end{document}
