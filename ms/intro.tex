\documentclass[11pt,article,oneside]{memoir}
\usepackage[]{org-preamble-pdflatex}
%% SJE mods ------------------------------------------------------------
%%\input{vc}
\usepackage{mathpazo}
%% SJE mods ------------------------------------------------------------



\title{\bigskip \bigskip Models of retinotopic development}

%\author{true}

\author{\Large Stephen J Eglen\vspace{0.05in} \newline\normalsize\emph{University of Cambridge} \newline\footnotesize \url{S.J.Eglen@damtp.cam.ac.uk}\vspace*{0.2in}\newline }

%\author{Stephen J Eglen (University of Cambridge)}

\date{}

\begin{document}  
\setkeys{Gin}{width=1\textwidth} 	
%\setromanfont[Mapping=tex-text,Numbers=OldStyle]{Minion Pro} 
%\setsansfont[Mapping=tex-text]{Minion Pro} 
%\setmonofont[Mapping=tex-text,Scale=0.8]{Pragmata}
\chapterstyle{article-4} 
\pagestyle{kjh}

%%\published{2015-01-19. Incomplete Draft. Please do not cite without permission.}
\published{2015-01-19}

\maketitle



\section{Introduction}\label{introduction}

How many types of mammalian retinal ganglion cell (RGC) are there? The
answer to this question depends partly on how you define a neuronal type
{[}Cook1998-bh{]}, but it is commonly assumed that RGC types have
distinct morphologies and physiologies. The pioneering work of Boycott
and Wassle (1974) suggested that there were at least three morphological
classes (alpha, beta and gamma) of RGC in cat, and these three types
mapped onto previously-defined physiological classes (X, Y and W)
{[}Cleland 1971, 1973{]}. For example, alpha cells were defined as
having larger dendritic fields and somata compared to neighbouring beta
cells. Since these early studies, subsequent work has primarily focused
on finer divisions of the gamma class which was thought to be a mixed
grouping (REF). Furthermore, it is unclear whether individual
morphological features alone are unique predictors of cell type, as
demonstrated by the large overlap in RGC somata area (their Figure 6)
among the alpha/beta/gamme cat RGCs, but that multiple features should
be considered simultaneously when classifying neurons. {[}Rodieck and
Brening 1983{]} formalised this notion, proposing to use multiple
features to define a multidimensional ``feature space'' in which to
define RGC types. If cells form distinct types, then the expectation is
that cells of the same type should cluster together in one part of this
feature space, and that different cell types occupy different parts of
feature space.

Recent advances in imaging and genetics have led to a dramatic increase
in data available, especially from mice but also other species, to
explore whether cells of distinct types form clusters in
multidimensional space. Estimates for mouse retina vary from 12
{[}Kong2005{]} to 22{[}Volgyi2009{]} based either on manual
classification of cell types to unsupervised approaches. These
unsupervised approaches use statistical methods to determine the optimal
number of clusters in the data (e.g.~using silhoutte widths technique;
REF). However, these approachees have no ground-truth data to compare
with the predicted number of cell types.

In this study, we analyse the morphology of RGCs from several mutant
mice lines where typically one or a few types of RGC is labelled with
GFP. We use supervised machine learning techniques to predict whether
the anatomical features can predict the ``genetic type'' of the mouse,
i.e.~the mouse line from where the cell was labelled. This provides us
with ground-truth data which we can use to evaluate our methods againts.
\textgreater{}From each RGC we measured fifteen features, from which we
found five that were highly predictive of cell type. We compare our
findings with a recent study (Sumbul2014) where near-perfect
classification was achieved when information about stratification depth
is included. We suggest that our anatomical measures can provide a
reliable basis for classification in the absence of stratification depth
information, and thus that the Brenning and Rodieck (1983) method of
classification is robust when applied to mouse RGCs.

\end{document}
